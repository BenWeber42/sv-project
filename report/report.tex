% This is based on "sig-alternate.tex" V1.9 April 2009
% This file should be compiled with V2.4 of "sig-alternate.cls" April 2009
%
\documentclass{report}

\usepackage[english]{babel}
\usepackage{graphicx}
\usepackage{tabularx}
\usepackage{subfigure}
\usepackage{enumitem}
\usepackage{url}

\usepackage{color}
\definecolor{orange}{rgb}{1,0.5,0}
\definecolor{lightgray}{rgb}{.9,.9,.9}
\definecolor{java_keyword}{rgb}{0.37, 0.08, 0.25}
\definecolor{java_string}{rgb}{0.06, 0.10, 0.98}
\definecolor{java_comment}{rgb}{0.12, 0.38, 0.18}
\definecolor{java_doc}{rgb}{0.25,0.35,0.75}

% code listings

\usepackage{listings}
\lstloadlanguages{Java}
\lstset{
	language=Java,
	basicstyle=\scriptsize\ttfamily,
	backgroundcolor=\color{lightgray},
	keywordstyle=\color{java_keyword}\bfseries,
	stringstyle=\color{java_string},
	commentstyle=\color{java_comment},
	morecomment=[s][\color{java_doc}]{/**}{*/},
	tabsize=2,
	showtabs=false,
	extendedchars=true,
	showstringspaces=false,
	showspaces=false,
	breaklines=true,
	numbers=left,
	numberstyle=\tiny,
	numbersep=6pt,
	xleftmargin=3pt,
	xrightmargin=3pt,
	framexleftmargin=3pt,
	framexrightmargin=3pt,
	captionpos=b
}

% Disable single lines at the start of a paragraph (Schusterjungen)

\clubpenalty = 10000

% Disable single lines at the end of a paragraph (Hurenkinder)

\widowpenalty = 10000
\displaywidowpenalty = 10000
 
% allows for colored, easy-to-find todos

\newcommand{\todo}[1]{\textsf{\textbf{\textcolor{orange}{[[#1]]}}}}

% consistent references: use these instead of \label and \ref

\newcommand{\lsec}[1]{\label{sec:#1}}
\newcommand{\lssec}[1]{\label{ssec:#1}}
\newcommand{\lfig}[1]{\label{fig:#1}}
\newcommand{\ltab}[1]{\label{tab:#1}}
\newcommand{\rsec}[1]{Section~\ref{sec:#1}}
\newcommand{\rssec}[1]{Section~\ref{ssec:#1}}
\newcommand{\rfig}[1]{Figure~\ref{fig:#1}}
\newcommand{\rtab}[1]{Table~\ref{tab:#1}}
\newcommand{\rlst}[1]{Listing~\ref{#1}}

% General information

\title{Software Verification -- Course Project Report}

% Use the \alignauthor commands to handle the names
% and affiliations for an 'aesthetic maximum' of six authors.

\numberofauthors{2}
\author{
% You can go ahead and credit any number of authors here,
% e.g. one 'row of three' or two rows (consisting of one row of three
% and a second row of one, two or three).
%
% The command \alignauthor (no curly braces needed) should
% precede each author name, affiliation/snail-mail address and
% e-mail address. Additionally, tag each line of
% affiliation/address with \affaddr, and tag the
% e-mail address with \email.
%
% 1st. author
\alignauthor Benjamin Weber\\
	\affaddr{ETH ID 11-933-017}\\
	\email{benweber@student.ethz.ch}
% 2nd. author
\alignauthor Marcel Mohler\\
	\affaddr{ETH ID 09-922-998}\\
	\email{mohlerm@student.ethz.ch}
}


\begin{document}

\maketitle

\begin{abstract}
\textbf{By using \textit{Software Verification} a programmer wants to achieve guarantees in terms of previously defined properties.\\
However, generally it is undecidable whether a given piece of code conforms to its specification, hence researchers have proposed and developed various semi-automatic methods.\\
One approach called \textit{Auto-Active Verification} tries to automate as much as possible, while the programmer provides guidance indirectly through program-level annotations\cite{tschannen2011verifying}\cite{autoactivelecture}.
We used the verifier AutoProof\cite{autoproof} to verify different algorithms on the program level. \\Furthermore, we present an implementation of quick sort\cite{quicksort} and bucket sort\cite{bucketsort} in the \textit{intermediate verification language} (IVL) Boogie\cite{barnett2006boogie}\cite{boogie}. We provide the needed annotations to formally prove correctness of the implementation with respect to the specification.
}
\end{abstract}

\section{Introduction}
\label{s:introduction}
As part of the \textbf{software verification} course offered by \textbf{ETH Zurich},
we had to complete a project involving software verification on the program
level with AutoProof as well as on the intermediate level through Boogie.\\
In section \ref{s:autoproof} we discuss software verification using AutoProof.
We continue to discuss the Boogie part in section \ref{s:boogie} and compare
the two in section \ref{s:comparison}. Finally, we present our conclusion in
section \ref{s:conclusion}.

\section{AutoProof verficiation}
\label{s:autoproof}

Many features already had a specification attached. Thus, it was sufficient to
make the AutoProof tool accept the implementation using hints such as assertions
or invariants. Features with incomplete specifications were more challenging as
they did not allow for such a high certainty regarding the correctness of the
solution.

\subsection{Verification of SV\_AUTOPROOF}

The \textbf{wipe} feature could successfully be verified once we specified in the loop
invariant that the postcondition was gradually established. This strategy
turned out to be useful for the whole project.
\\
\\
The \textbf{mod\_three} feature could be verified similarly. We just had to
specify what did not get modified additionally.
\\
\\
We needed to specify more to verify the \textbf{swapper} feature. To establish
the preconditions of \textbf{swap} two invariants regarding \textbf{x} and
\textbf{y} were necessary. Once those preconditions could be established, the
previous strategy was sufficient to verify the whole feature.
\\
\\
The \textbf{search} feature is special because we also had to specify the 
postconditions ourselves. No tool can verify whether the postconditions truly
capture the full extent of what the programmer means to achieve.\\
We believe the correct meaning of the \textbf{search} feature is that it does
not modify the list to be searched and that it returns \textbf{true} if and only if
the given integer is in that list.\\
The \textbf{has} feature of the ghost-feature \textbf{sequence} was very useful
to express the intuition above in AutoProof.\\
To verify the feature it was sufficient to explicitly state the implications
of \textbf{Result} for both cases \textbf{true} and \textbf{false}.
\\
\\
Verifying the \textbf{prod\_sum} feature was rather trivial by specifying that
the postcondition was gradually established.
\\
\\
In contrast the \textbf{paly} feature was more complicated. First some preconditions
had to be established by adding simple invariants about \textbf{x} and \textbf{y}.
Then we used the strategy to show AutoProof that the postcondition is gradually
established by the loop.\\
But that still was not enough. Thus we had to specify explicitly the counter-example
when \textbf{Result} became \textbf{false}.\\
Again, we had to be careful that the postcondition fully captured the meaning
of the \textbf{paly} feature. We believe we achieved this by establishing the
correctness through the postcondition involving the result and by establishing
that the parameter does not get modified.

\subsection{Experience using AutoProof}

It is noteworthy that AutoProof's modify clause is very useful for specification.
Because every specification should technically contain what does not get modified
in addition to describing what gets modified. Due to the modify clause this
becomes much easier, because AutoProof can simply assume that everything not part
of the modify clause stays unmodified.
\\
\\
Furthermore, the ghost features of \textbf{SIMPLE\_ARRAY} were very useful to
reason about sequences and it seems that AutoProof can handle them pretty well.

\section{Boogie}
\label{s:boogie}
Boogie is an intermediate verification language (IVL), intended as a layer on which to build program verifiers for other languages\cite{boogiegithub}.\\
It is also the name of a tool that accepts Boogie language as input and then generates verification conditions that are passed to an SMT solver such as Z3\cite{de2008z3}\cite{z3}.\\
Initially, it has been developed by Microsoft Research but it is also open to other contributors on GitHub now.

\subsection{Boogie implementation}

Using Boogie to implement the given algorithms did not pose a challenge.
Even though Boogie's simplicity, both algorithms could be expressed well.
While Boogie has powerful means to abstract complicated concepts, we only
needed some very simple language structures of Boogie to express an array to be sorted.\\
One particularly interesting design choice was to implement the partition step
of the quick sort algorithms in such a way that it could be reused by the bucket sort algorithm.\\
This simplified the implementation due to the additional code reuse and did not
make the verification or specification more complicated.

\todo{Implement the sorting algorithm described in
Section 3.2. Discuss your design choices, and how you modeled the algorithm using the
primitives of Boogie}

\subsection{Boogie specification}
In principle, our sort implementations have to fulfill two properties.\\
Most importantly, the resulting array has to be sorted. This is pretty trivial and we used
a simple Boolean function to express this specification.
\begin{lstlisting}
function sorted(a: [int]int, lo, hi: int): bool
{
  hi - lo >= 1
  &&
  (forall i, j: int :: lo <= i 
  && i <= j 
  && j <= hi ==> a[i] <= a[j]
  )
}
\end{lstlisting}
Furthermore we need to make sure, that the sorted array is actually a valid
permutation of the input array. Modeling this property turned out to be the most difficult part
of this project.\\
The art is to find a way to express the permutation property in such a way that
Boogie can handle it well and that it is also useful to work with for verification
and specification.\\
As such we tried several different solutions which we will describe in more detail
in the following.\\
\\
\\
A very early idea was to define the properties with axioms.
\begin{lstlisting}
// uninterpreted function describing permutations
function permutation(a, b: [int]int): bool;

// reflexivity
axiom (forall a: [int]int :: permutation(a, a));
// symmetry
axiom (forall a, b: [int]int :: permutation(a, b)
  ==> permutation(b, a));
// transitivity
axiom (forall a, b, c: [int]int ::
  permutation(a, b) && permutation(b, c)
    ==> permutation(a, c));
\end{lstlisting}
This allows one to easily introduce trivial properties of permutations such as
reflexivity, symmetry and transitivity. Once, one can proof that swap preserves
the permutation property, Boogie can infer that whole procedures also preserve
it, if they only use the swap procedure to modify the input array (which can be easily achieved).\\
While this is an attractive approach as most of the complexity is contained in the swap procedure,
it is also dangerous in general. The problem with axioms and assumptions as well
in Boogie is that Boogie will assume them to be true unconditionally. Therefore,
one has to be very careful not to introduce unsound axioms or assumptions.\\
In a sense, one loses some of the certainty gained by using sound tools such as
Boogie, because correctness additionally relies on the correctness and soundness
of the axioms and assumptions.\\
In this specific case it may be reasonable to use such trivial axioms. A more
interesting approach is to use Boogie to proof these axioms automatically. Using our
definition of \textbf{perm\_of} one could use only the following axioms to
proof these properties of permutations:
\begin{lstlisting}
axiom (forall a, b: [int]int :: permutation(a, b)
   ==> (exists m: [int]int :: perm_of(a, b, m)));
axiom (forall a, b, m: [int]int :: perm_of(a, b, m)
   ==> permutation(a, b));
\end{lstlisting}

And then one could use a procedure to give a constructive proof for the
reflexivity of permutations:
\begin{lstlisting}
procedure reflexivity(a: [int]int)
  ensures permutation(a, a);
{
   var m: [int]int;
   var k: int;

   k := 0;
   while (k < N)
      invariant 0 <= k && k <= N;
      invariant (forall i: int :: 0 <= i && i < k ==> m[i] == i);
   {
      m[k] := k;
      k := k + 1;
   }
   assert perm_of(a, a, m);
}
\end{lstlisting}

Then this gives strong reason to the legitimacy of using the reflexivity axiom
during verification. For symmetry one would have to make the following procedure
verify:
\begin{lstlisting}
procedure symmetry(a, b: [int]int)
   requires permutation(a, b);
   ensures permutation(b, a);
{
   var m, _m: [int]int;
   // this is allowed due to the axioms
   // linking permutation and perm_of
   assume perm_of(a, b, m);
                 
   /* constructive proof */
                     
   assert perm_of(b, a, _m);
}
\end{lstlisting}

And for transitivity one could use a similar approach as used for symmetry.\\
However, we were not able to verify the symmetry and transitivity property this way.
Furthermore, Boogie might not be the right tool for such an approach.
Nevertheless, it was an interesting excursion into another aspect of verification.
\\
\\
Another idea which we explored was to look at the histograms of two arrays to
check whether they were permutations of each other. One can (recursively) define a
Boogie function \textbf{count} which determines how often a certain element appears in an 
array. If every element of one array has the same count in another array, then they
are permutations of each other.\\
This has the advantage that no permutation array is required for verification.\\
However, it turned out that Boogie can not handle these counts well and we were
unable to verify the permutation property with this approach.
\\
\\
Because of the mentioned pitfalls of the previous attempts we decided to model the property by keeping an additional permutation array, that stores the modification from the input array to the final, sorted array. This allows us to easily verify that the current array is a permutation of the initial one. Further details can be found in Section \ref{s:boogie_verification}.\todo{wieso hemmer eigentlech die init() methode nuemm?}
\\\\
In general, expressing the specifications on Boogie was interesting due to two different aspects.
\\
\\
Firstly, there was the challenge to fully capture the meaning of an algorithm
through the specification.\\
Boogie was very helpful there with its rather powerful expression language which
contains both the universal and existential quantifiers.
However, one also had to be careful to express formulas such that Boogie could
handle them well. Otherwise, more annotations are necessary which is less desirable.
\\
\\
Secondly, it was often surprising to see what a complete specification was for a procedure.
An incomplete specification can lead to the failure of the verification.
Thus we often came across incomplete specifications.\\
We learned that in addition to specifying what did get changed we also had to
specify what did not get changed. A complicated example of this can be found in
the \textbf{quick sort} and \textbf{partition} procedures.\\
Both of these procedures preserve the property that if some values were previously
bigger-equal than all values in a given range, those values will still be bigger-equal
after the procedure call.\\
While this is trivially obvious because both procedures only permute an array,
it is not trivially obvious that this property is necessary to verify the
\textbf{quick sort} and \textbf{bucket sort} implementations.

\todo{Specify the complete behavior of the sorting algorithm
using the specification primitives of Boogie. Discuss your specification choices, in
particular, how you modeled the “permutation” property for the resulting array.
Describe any difficulties and how you overcame them. Contrast the specification
language of Boogie with the specification language of AutoProof.}

\subsection{Boogie verification}
\label{s:boogie_verification}
We were able to verify the correctness of our implementation based on the sorted and permutation properties.\\
We will describe a few challenges and issues we encountered during the verification process in the following.
\\\\
One of the main issues we had was termination. In Boogie one can only proove partial correctness, i.e. correctness only for the program traces that do terminate.
Boogie ignores program traces that do not terminate. It would have been very useful to be able to proove total correctness which means that the program is
guaranteed to terminate for all input parameters in addition conforming to its specification.\\
Applied to our scenario this meant that if we do not carefully design our algorithms such that they always terminate, we might end up with postconditions that evaluate to trivially true because they can never be reached.\\
See this small example, which verifies correctly in Boogie even though \texttt{1 == 0} is clearly a false statement.
\begin{lstlisting}
procedure F(n: int) returns (r: int)
  ensures 1 == 0;
{
  while(true) {
    r := n;
  }
  r := 0;
}
\end{lstlisting}
If we change the \texttt{while(true)} in line 4 to \texttt{while(false)} Boogie correctly reports that the postcondition might not hold.
\\\\
We also found the fact that we can not define pre-conditions for functions to be rather limiting. Taking the example of \texttt{sorted()}, we would have liked to require $N \geq 2$ because sortedness does not make sense for lists of lenght 1 or less. In our implementation we decided to conservatively define \textbf{sorted} to be false for those cases. However, this definition can still yield false positives if the expression \textbf{!sorted(...)} is used. A precondition would have allowed us to handle this better. This is similar to division which is also undefined for divisor being zero.
\\
Additionally, this forces one to think of the cases where $N < 2$ and it made us realize that we should specify the behaviour of our implementation also for this case. We believe in this case a sorting function should leave the array untouched. If one does not specify this, a sorting function could be free to do anything with the input array if its length is 1 or less!
\\\\
Another observation is the difference between defining a pre- and postcondition and defining a single postcondition which includes an implication.
An example is drawn below:
\begin{lstlisting}
procedure quickSort_a(lo, hi: int)
  ensures perm_of(old(a), old(old_a), old(perm)) 
  ==> perm_of(a, old_a, perm);
{..}
procedure quickSort_b(lo, hi: int)
  requires perm_of(a, old_a, perm);
  ensures perm_of(a, old_a, perm);
{..}
\end{lstlisting}
Option a is better (and we used this in our implementation) because preconditions should be requirements of the algorithm to work correctly.
However, quicksort works correctly without \textbf{perm\_of(a, old\_a, perm)}.
\\
This way the algorithm can be used without having to establish the permutation property
and instead it can be voluntarily established, if one wishes to make use of it.
\\\\
One interesting thing we noticed with pre-conditions is that they lead to ad-hoc code.
For example the pre-condition $lo <= hi$ of the \textbf{quicksort} procedure leads to
an additional \textbf{if} clause for every single call of \textbf{quicksort}.\\
The only exception being in the \textbf{sort} procedure, because it implicitelly
establishes this pre-condition. We are unsure whether this side-effect was
introduced deliberatelly or not. The following assertion can be verified by Boogie:
\begin{lstlisting}
assert !has_small_elements(a) ==> N > 0;
\end{lstlisting}
However, with regards to the ad-hoc code due to pre-conditions, one might consider
to remove these pre-conditions and simply deal with those cases separately inside
the procedure. Though, we believe this solution is not optimal, because one can not
be sure that the procedure handles these cases correctly in all situations.\\
Thus, we see this ad-hoc code as a cost, to ensure that these cases are handled
correctly (in our case it is just a coincidence that the when $lo > hi$ \textbf{quicksort}
should simply do nothing for every \textbf{quicksort} call).
\todo{loop invariants vs assertions, dass halt loop invariante wie pre-/post-conditions send (stoht vl au em language manual so ähnlech)}
\\\\
\todo{Verify your Boogie program using Boogie. Report any
significant problems you encountered; for example, which procedures you could verify
and which ones you could not. Describe if there were any aspects of the implementation
or of the specification you had to change to make them easier to verify. Describe which parts of the specification you could not verify, and what the limitations were that
prevented you from doing it. Explain how you achieved modular verification.}

Please note, that we also carefully annotated the attached code with many comments and explanations which exceed the scope of this report.

\section{Comparison AutoProof \& Boogie}
\label{s:comparison}
If we talk about Boogie as a verification language (not the tool) AutoProof is one layer above Boogie as it compiles into Boogie code.
\\
AutoProof works on a real programming language and can be applied to complex programs more easily. Especially if we would consider complex object structures being able to model the program in Eiffel is a lot easier than only using the rather primitive structures of Boogie.
\todo{was gits det gross, zsaege?}
\\
\todo{Particular effort should be made to contrast your experiences doing verification at the
program level (AutoProof) and intermediate verification language level (Boogie).}

\section{Conclusion}
\label{s:conclusion}
We found it very exciting to apply the concepts of Auto-Active Verification to a larger example. In the lectures we have always seen these concepts being applied to small and simple functions, where verification was rather trivial. And while quick sort and bucket sort are still only a few dozen lines of code, the verification code adds at least the same amount of lines on top of the implementation.
We realized that it can be tricky to design a sound system...
\todo{finish}



% The following two commands are all you need in the
% initial runs of your .tex file to
% produce the bibliography for the citations in your paper.
\bibliographystyle{abbrv}
\bibliography{report}  % sigproc.bib is the name of the Bibliography in this case
% You must have a proper ".bib" file

%\balancecolumns % GM June 2007

\end{document}
