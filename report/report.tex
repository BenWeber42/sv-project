% This is based on "sig-alternate.tex" V1.9 April 2009
% This file should be compiled with V2.4 of "sig-alternate.cls" April 2009
%
\documentclass{report}

\usepackage[english]{babel}
\usepackage{graphicx}
\usepackage{tabularx}
\usepackage{subfigure}
\usepackage{enumitem}
\usepackage{url}

\usepackage{color}
\definecolor{orange}{rgb}{1,0.5,0}
\definecolor{lightgray}{rgb}{.9,.9,.9}
\definecolor{java_keyword}{rgb}{0.37, 0.08, 0.25}
\definecolor{java_string}{rgb}{0.06, 0.10, 0.98}
\definecolor{java_comment}{rgb}{0.12, 0.38, 0.18}
\definecolor{java_doc}{rgb}{0.25,0.35,0.75}

% code listings

\usepackage{listings}
\lstloadlanguages{Java}
\lstset{
	language=Java,
	basicstyle=\scriptsize\ttfamily,
	backgroundcolor=\color{lightgray},
	keywordstyle=\color{java_keyword}\bfseries,
	stringstyle=\color{java_string},
	commentstyle=\color{java_comment},
	morecomment=[s][\color{java_doc}]{/**}{*/},
	tabsize=2,
	showtabs=false,
	extendedchars=true,
	showstringspaces=false,
	showspaces=false,
	breaklines=true,
	numbers=left,
	numberstyle=\tiny,
	numbersep=6pt,
	xleftmargin=3pt,
	xrightmargin=3pt,
	framexleftmargin=3pt,
	framexrightmargin=3pt,
	captionpos=b
}

% Disable single lines at the start of a paragraph (Schusterjungen)

\clubpenalty = 10000

% Disable single lines at the end of a paragraph (Hurenkinder)

\widowpenalty = 10000
\displaywidowpenalty = 10000
 
% allows for colored, easy-to-find todos

\newcommand{\todo}[1]{\textsf{\textbf{\textcolor{orange}{[[#1]]}}}}

% consistent references: use these instead of \label and \ref

\newcommand{\lsec}[1]{\label{sec:#1}}
\newcommand{\lssec}[1]{\label{ssec:#1}}
\newcommand{\lfig}[1]{\label{fig:#1}}
\newcommand{\ltab}[1]{\label{tab:#1}}
\newcommand{\rsec}[1]{Section~\ref{sec:#1}}
\newcommand{\rssec}[1]{Section~\ref{ssec:#1}}
\newcommand{\rfig}[1]{Figure~\ref{fig:#1}}
\newcommand{\rtab}[1]{Table~\ref{tab:#1}}
\newcommand{\rlst}[1]{Listing~\ref{#1}}

% General information

\title{Software Verification -- Course Project Report}

% Use the \alignauthor commands to handle the names
% and affiliations for an 'aesthetic maximum' of six authors.

\numberofauthors{2}
\author{
% You can go ahead and credit any number of authors here,
% e.g. one 'row of three' or two rows (consisting of one row of three
% and a second row of one, two or three).
%
% The command \alignauthor (no curly braces needed) should
% precede each author name, affiliation/snail-mail address and
% e-mail address. Additionally, tag each line of
% affiliation/address with \affaddr, and tag the
% e-mail address with \email.
%
% 1st. author
\alignauthor Benjamin Weber\\
	\affaddr{ETH ID }\\
	\email{benweber@student.ethz.ch}
% 2nd. author
\alignauthor Marcel Mohler\\
	\affaddr{ETH ID 09-922-998}\\
	\email{mohlerm@student.ethz.ch}
}


\begin{document}

\maketitle

\begin{abstract}
\textbf{By using \textit{Software Verfification} the programmer wants to achieve guarantees in terms of previously defined properties.\\
In theory however, the underlaying problem is undecidable hence researchers have proposed and developed various semi-automatic methods.\\
One approach called \textit{Auto-Active Verification} tries to automate as much as possible, while the programmer provides guidance indirectly through program-level annotations\cite{autoactivelecture}.
We use the verifier AutoProof\cite{autoproof} to verify an existing simple list data structure on the program level. \\Furthermore, we present an implementation of quick sort\cite{quicksort} and bucket sort\cite{bucketsort} in the \textit{intermediate verification language} (IVL) boogie\cite{boogie}. We provide the needed annotations to formally prove correctness of the implementation in respect to the specification.
}
\end{abstract}

\section{Introduction}
\todo{is this really needed?}

\section{Autoproof verficiation}
\todo{Add missing specifications to all features of the
SV\_AUTOPROOF class as embedded contracts (without altering the implementation).
Describe how you were able to specify complete specifications, and any problems that
occurred. ​
Using AutoProof, verify as many features of SV\_AUTOPROOF as possible.
Discuss if there were any aspects of the specification you had to change to make them
easier to verify. Describe which parts of the specification you could not verify, and what
the limitations were that prevented you from doing it.}

\section{Boogie}


\subsection{Boogie implementation}
\todo{Implement the sorting algorithm described in
Section 3.2. Discuss your design choices, and how you modelled the algorithm using the
primitives of Boogie}

\subsection{Boogie specification}
\todo{Specify the complete behaviour of the sorting algorithm
using the specification primitives of Boogie. Discuss your specification choices, in
particular, how you modelled the “permutation” property for the resulting array.
Describe any difficulties and how you overcame them. Contrast the specification
language of Boogie with the specification language of AutoProof.}

\subsection{Boogie verification}
\todo{Verify your Boogie program using Boogie. Report any
significant problems you encountered; for example, which procedures you could verify
and which ones you could not. Describe if there were any aspects of the implementation
or of the specification you had to change to make them easier to verify. Describe whichparts of the specification you could not verify, and what the limitations were that
prevented you from doing it. Explain how you achieved modular verification.}

\subsection{Comparison Autoproof \& Boogie}
\todo{Particular effort should be made to contrast your experiences doing verification at the
program level (AutoProof) and intermediate verification language level (Boogie).}

\section{Conclusion}
\todo{CONCLUDE}



% The following two commands are all you need in the
% initial runs of your .tex file to
% produce the bibliography for the citations in your paper.
\bibliographystyle{abbrv}
\bibliography{report}  % sigproc.bib is the name of the Bibliography in this case
% You must have a proper ".bib" file

%\balancecolumns % GM June 2007

\end{document}
