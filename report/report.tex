% This is based on "sig-alternate.tex" V1.9 April 2009
% This file should be compiled with V2.4 of "sig-alternate.cls" April 2009
%
\documentclass{report}

\usepackage[english]{babel}
\usepackage{graphicx}
\usepackage{tabularx}
\usepackage{subfigure}
\usepackage{enumitem}
\usepackage{url}

\usepackage{color}
\definecolor{orange}{rgb}{1,0.5,0}
\definecolor{lightgray}{rgb}{.9,.9,.9}
\definecolor{java_keyword}{rgb}{0.37, 0.08, 0.25}
\definecolor{java_string}{rgb}{0.06, 0.10, 0.98}
\definecolor{java_comment}{rgb}{0.12, 0.38, 0.18}
\definecolor{java_doc}{rgb}{0.25,0.35,0.75}

% code listings

\usepackage{listings}
\lstloadlanguages{Java}
\lstset{
	language=Java,
	basicstyle=\scriptsize\ttfamily,
	backgroundcolor=\color{lightgray},
	keywordstyle=\color{java_keyword}\bfseries,
	stringstyle=\color{java_string},
	commentstyle=\color{java_comment},
	morecomment=[s][\color{java_doc}]{/**}{*/},
	tabsize=2,
	showtabs=false,
	extendedchars=true,
	showstringspaces=false,
	showspaces=false,
	breaklines=true,
	numbers=left,
	numberstyle=\tiny,
	numbersep=6pt,
	xleftmargin=3pt,
	xrightmargin=3pt,
	framexleftmargin=3pt,
	framexrightmargin=3pt,
	captionpos=b
}

% Disable single lines at the start of a paragraph (Schusterjungen)

\clubpenalty = 10000

% Disable single lines at the end of a paragraph (Hurenkinder)

\widowpenalty = 10000
\displaywidowpenalty = 10000
 
% allows for colored, easy-to-find todos

\newcommand{\todo}[1]{\textsf{\textbf{\textcolor{orange}{[[#1]]}}}}

% consistent references: use these instead of \label and \ref

\newcommand{\lsec}[1]{\label{sec:#1}}
\newcommand{\lssec}[1]{\label{ssec:#1}}
\newcommand{\lfig}[1]{\label{fig:#1}}
\newcommand{\ltab}[1]{\label{tab:#1}}
\newcommand{\rsec}[1]{Section~\ref{sec:#1}}
\newcommand{\rssec}[1]{Section~\ref{ssec:#1}}
\newcommand{\rfig}[1]{Figure~\ref{fig:#1}}
\newcommand{\rtab}[1]{Table~\ref{tab:#1}}
\newcommand{\rlst}[1]{Listing~\ref{#1}}

% General information

\title{Software Verification -- Course Project Report}

% Use the \alignauthor commands to handle the names
% and affiliations for an 'aesthetic maximum' of six authors.

\numberofauthors{2}
\author{
% You can go ahead and credit any number of authors here,
% e.g. one 'row of three' or two rows (consisting of one row of three
% and a second row of one, two or three).
%
% The command \alignauthor (no curly braces needed) should
% precede each author name, affiliation/snail-mail address and
% e-mail address. Additionally, tag each line of
% affiliation/address with \affaddr, and tag the
% e-mail address with \email.
%
% 1st. author
\alignauthor Benjamin Weber\\
	\affaddr{ETH ID 11-933-017}\\
	\email{benweber@student.ethz.ch}
% 2nd. author
\alignauthor Marcel Mohler\\
	\affaddr{ETH ID 09-922-998}\\
	\email{mohlerm@student.ethz.ch}
}


\begin{document}

\maketitle

\begin{abstract}
\textbf{By using \textit{Software Verification} a programmer wants to achieve guarantees in terms of previously defined properties.\\
In theory however, it is undecidable whether a given piece of code conforms to its specification, hence researchers have proposed and developed various semi-automatic methods.\\
One approach called \textit{Auto-Active Verification} tries to automate as much as possible, while the programmer provides guidance indirectly through program-level annotations\cite{autoactivelecture}.
We use the verifier AutoProof\cite{autoproof} to verify different algorithms on the program level. \\Furthermore, we present an implementation of quick sort\cite{quicksort} and bucket sort\cite{bucketsort} in the \textit{intermediate verification language} (IVL) boogie\cite{boogie}. We provide the needed annotations to formally prove correctness of the implementation with respect to the specification.
}
\end{abstract}

\section{Introduction}
As part of the \textbf{software verification} course offered by \textbf{ETHZ},
we had to complete a project involving software verification on the program
level with AutoProof as well as on the intermediate level through Boogie.

In section \ref{s:autoproof} we discuss software verification using AutoProof.
We continue to discuss the Boogie part in section \ref{s:boogie} and compare
the two in section \ref{s:comparison}. Finally, we present our conclusion in
section \ref{s:conclusion}.

\section{Autoproof verficiation}
\label{s:autoproof}

Many features already had a specification attached. Thus, it was sufficient to
make the autoproof tool accept the implementation using hints such as assertions
and invariants.
\\
\\
The \textbf{wipe} feature could successfully be verified once we specified in the loop
invariant that the post-condition was gradually established. This strategy
turned out to be useful for the whole project.
\\
\\
The \textbf{mod\_three} feature could be verified similarly. We just had to
specify what didn't get modified additionally.
\\
\\
We needed to specify more to verify the \textbf{swapper} feature. To establish
the pre-conditions of \textbf{swap} two invariants regarding \textbf{x} and
\textbf{y} were necessary. Once those pre-conditions could be established, the
previous strategy was sufficient to verify the whole feature.
\\
\\
The \textbf{search} feature is special because we also had to specify the 
post-conditions ourselves. No tool can verify whether the post-conditions truely
capture the full extent of what the programmer means to achieve.

We believe the correct meaning of the \textbf{search} feature is that it does
not modify the list to be search and that it returns \textbf{true} only iff
the given integer is in that list.

The \textbf{has} feature of the ghost-feature \textbf{sequence} was very useful
to express the intuition above in AutoProof.

To verify the feature it was sufficient to explicitelly state the implications
of \textbf{Result} for both cases \textbf{true} and \textbf{false}.
\\
\\
Verifying the \textbf{prod\_sum} feature was rather trivial by specifying that
the post-condition was gradually established.
\\
\\
In contrast the \textbf{paly} feature was more complicated. First the pre-conditions
had to be established by adding simple invariants about \textbf{x} and \textbf{y}.
Then we used the strategy to show AutoProof that the post-condition is gradually
established by the loop.

But that still wasn't enough. Thus we had to specify explicitelly the counter-example
when \textbf{Result} became \textbf{false}.

Again, we had to be careful that the post-condition fully captured the meaning
of the \textbf{paly} feature. We believe we achieved this by establishing the
correctness through the post-condition involving the result and by establishing
that the parameter does not get modified.



\todo{Add missing specifications to all features of the
SV\_AUTOPROOF class as embedded contracts (without altering the implementation).
Describe how you were able to specify complete specifications, and any problems that
occurred. ​
Using AutoProof, verify as many features of SV\_AUTOPROOF as possible.
Discuss if there were any aspects of the specification you had to change to make them
easier to verify. Describe which parts of the specification you could not verify, and what
the limitations were that prevented you from doing it.}

\section{Boogie}
\label{s:boogie}

\subsection{Boogie implementation}
\todo{Implement the sorting algorithm described in
Section 3.2. Discuss your design choices, and how you modelled the algorithm using the
primitives of Boogie}

\subsection{Boogie specification}
\todo{Specify the complete behaviour of the sorting algorithm
using the specification primitives of Boogie. Discuss your specification choices, in
particular, how you modelled the “permutation” property for the resulting array.
Describe any difficulties and how you overcame them. Contrast the specification
language of Boogie with the specification language of AutoProof.}

\subsection{Boogie verification}
\todo{Verify your Boogie program using Boogie. Report any
significant problems you encountered; for example, which procedures you could verify
and which ones you could not. Describe if there were any aspects of the implementation
or of the specification you had to change to make them easier to verify. Describe whichparts of the specification you could not verify, and what the limitations were that
prevented you from doing it. Explain how you achieved modular verification.}

\section{Comparison Autoproof \& Boogie}
\label{s:comparison}
\todo{Particular effort should be made to contrast your experiences doing verification at the
program level (AutoProof) and intermediate verification language level (Boogie).}

\section{Conclusion}
\label{s:conclusion}
\todo{CONCLUDE}



% The following two commands are all you need in the
% initial runs of your .tex file to
% produce the bibliography for the citations in your paper.
\bibliographystyle{abbrv}
\bibliography{report}  % sigproc.bib is the name of the Bibliography in this case
% You must have a proper ".bib" file

%\balancecolumns % GM June 2007

\end{document}
